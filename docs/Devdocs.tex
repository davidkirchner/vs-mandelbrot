\documentclass{article}

\usepackage[margin=3cm]{geometry}
\usepackage{amsmath}
\usepackage{amsfonts}
\usepackage{amssymb}
\usepackage{amscd}
\usepackage{standalone}
\usepackage{float}
\usepackage{color}
\usepackage[shortlabels]{enumitem}
\usepackage{graphicx}
\usepackage{caption}
\usepackage[ngerman]{babel}
\usepackage{lscape}
\usepackage{cancel}
\usepackage{	}
\usepackage{listings}

\definecolor{codegreen}{rgb}{0,0.6,0}
\definecolor{codegray}{rgb}{0.5,0.5,0.5}
\definecolor{codepurple}{rgb}{0.58,0,0.82}
\definecolor{backcolour}{rgb}{0.95,0.95,0.92}
\lstdefinestyle{mystyle}{
    backgroundcolor=\color{backcolour},   
    commentstyle=\color{codegreen},
    keywordstyle=\color{magenta},
    numberstyle=\tiny\color{codegray},
    stringstyle=\color{codepurple},
    basicstyle=\ttfamily\footnotesize,
    breakatwhitespace=false,         
    breaklines=true,                 
    captionpos=b,                    
    keepspaces=true,                 
    numbers=left,                    
    numbersep=5pt,                  
    showspaces=false,                
    showstringspaces=false,
    showtabs=false,                  
    tabsize=2
}

\lstset{style=mystyle, language = java}

\begin{document}
\begin{titlepage}
    \centering
    {\scshape\LARGE Hochschule für Technik und Wirtschaft Dresden \par}
    \vspace{1cm}
    {\scshape\Large Berechnung der Mandelbrotmenge \\mit Implementierung von Java-RMI \par}
    \vspace{1.5cm}
    {\huge\bfseries Development Document\par}
    \vspace{2cm}
    {\Large\itshape David Kirchner, Maxim Haschke, Tan Minh Ho, Quang Duy Pham\par}
    \vfill

    {\large \today\par}
\end{titlepage}

\tableofcontents

\newpage
\section{Kurzbeschreibung an Belegarbeit}
	Um das Modul Programmierung verteilter Systeme zu erfassen, wird eine Gruppenarbeit zur Entwicklung eines verteilten Programms zur Anzeige der Mandelbrotmenge belegt:
	\begin{itemize}
		\item schrittweiser Zoom zu einem bestimmten Bildpunkt
		\item Nutzung von Java-RMI oder einer anderen geeigneten Technologie zur Verteilung der Aufgaben
		\item Architektur Model-View-Presenter für den Client
		\item Dokumentation der Programmstruktur
	\end{itemize}

\section{Architektur}
	Nach der Diskussion mit anderen Gruppenteilnehmer wird das Programm mit folgenden Merkmalen realisiert:
	\begin{itemize}
		\item Übertragung mittels TCP
		\item Nutzung von 2-dimensionalem Array zur Übertragung der Pixelfarbe
		\item Thread-Pool im Client zum Anzeigen des berechneten Bildes
		\item Verwendung von Threads im Server zur Berechnung der Mandelbrotmenge
	\end{itemize}
	
\section{Details der Programme}
\subsection{Client}
	Fuer den Client gibt es drei Java-Datein mit folgenden Bezeichnungen:
	\begin{itemize}
		\item MandelClient.java
		\item MandelClientImpl.java
		\item MandelClientMain.java
	\end{itemize}
\subsubsection{MandelClient.java}
	Die Datei \glqq MandelClient.java\grqq definiert die Schnittstelle fuer das Clientprogramm. Die Methoden setRGB, sendTasks und setZoomDestination sind hier als Remote mit RemoteException deklariert.
	\lstinputlisting[]{../src/MandelClient.java}

	\newpage
\subsubsection{MandelClientImpl.java}
	Die Datei \glqq MandelClientImpl.java\grqq implementiert oben genannte Methoden fuer den Client.\\

	\begin{enumerate}
		\item Konstruktor setzt die notwendigen Bildwerte zum Server und setzt JFrame und ActionListener fest.\\
			\lstinputlisting[firstline=36, lastline=84]{../src/MandelClientImpl.java}
			\newpage
		\item sendTasks schickt Request zum Server und wartet auf Antwort\\
			\begin{enumerate}
				\item Senden der Koordinaten und Werte von ActionListener zum Server
				\item Server startet Berechnung der Mandelbrotmenge mit Zoom zu den gegebenen Koordinaten
				\item Client wartet auf Server
				\item Client zeigt die zurueckgegeben RGB-Werte von Server an
			\end{enumerate}
			\lstinputlisting[firstline=88, lastline=102]{../src/MandelClientImpl.java}
		\item setRGB setzt Pixelfarbe des Bildes mit Nutzung eines Thread-Pools.
			\lstinputlisting[firstline=165, lastline=175]{../src/MandelClientImpl.java}
	\end{enumerate}
	
	\newpage
	\subsubsection{MandelClientMain.java}
	Die Datei \glqq MandelClientMain.java\grqq sucht den Server mit bestimmtem Namen (welcher durch args[]-Variable gesetzt wurde) und schickt dann Request zum Server.
	\lstinputlisting[]{../src/MandelClientMain.java}

	\newpage
	\subsection{Server}
	Fuer den Client gibt es drei Javadateien mit Name:
	\begin{itemize}
		\item MandelServer.java
		\item MandelServerImpl.java
		\item MandelServerMain.java
	\end{itemize}
	
	\subsubsection{MandelServer.java}
	Die Datei \glqq MandelServer.java\grqq definiert die Schnittstelle fuer das Serverprogramm. Die Methoden setDetail, setImageProperties, calculateRGB, returnColor, startCalculatingRGB und isFinish sind hier als Remote mit RemoteException deklariert.
	\lstinputlisting[]{../src/MandelServer.java}
	
	\newpage	
	\subsubsection{MandelServerImpl.java}
	\begin{enumerate}
		\item Konstruktor definiert Thread-Pool fuer Parallelität des Servers
			\lstinputlisting[firstline=22, lastline=40]{../src/MandelServerImpl.java}
		\item setDetails definiert die Laenge, Breite und Details (Iterationen) des Bildern und deklariert ein 2-dimensionales Array mit passender Laenge und Breite.
			\lstinputlisting[firstline=61, lastline=66]{../src/MandelServerImpl.java}
		\item setImageProperties setzt die Koordinaten von MouseActionListener.
			\lstinputlisting[firstline=68, lastline=72]{../src/MandelServerImpl.java}
		\item returnColor schickt ein 2-dimensionales Array zum Client zurueck
			\lstinputlisting[firstline=74, lastline=76]{../src/MandelServerImpl.java}
		\newpage		
		\item startCalculateingRBG schnitt das Bild zu kleineren Bilden mit Hilfe der Nummer des Threads im Thread-Pool. Falls alle Threads "busy" sind, wartet Segment des Bildes 
			\lstinputlisting[firstline=78, lastline=96]{../src/MandelServerImpl.java}
		\item isFinish beanwortet, ob Bild fertig bearbeitet hat
			\lstinputlisting[firstline=98, lastline=106]{../src/MandelServerImpl.java}
		\item calculateRGB stellt Aufgabe zum Thread fest, um parallele Berechnung zu erreichen
			\lstinputlisting[firstline=108, lastline=111]{../src/MandelServerImpl.java}
		\newpage
		\item Task berechnet die Mandelbrotmenge
			\lstinputlisting[firstline=113, lastline=149]{../src/MandelServerImpl.java}
	\end{enumerate}
	
	\subsubsection{MandelServerMain.java}
	Die Datei \glqq MandelServerMain.java\grqq stellt ein Registry fest und verbindet mit Serverschnittstelle
	\lstinputlisting[]{../src/MandelServerMain.java}
	
	\newpage
	\subsection{Link und Literatur}
	\begin{itemize}
		\item Parallele und verteilte Anwendungen in Java, Rainer Oechsle
		\item Berechnung der Mandelbrotmenge: https://www.youtube.com/watch?v=0bgrzmtu4e8
	\end{itemize}
\end{document}


