\documentclass{article}

\usepackage[margin=3cm]{geometry}
\usepackage{amsmath}
\usepackage{amsfonts}
\usepackage{amssymb}
\usepackage{amscd}
\usepackage{standalone}
\usepackage{float}
\usepackage{color}
\usepackage[shortlabels]{enumitem}
\usepackage{graphicx}
\usepackage{caption}
\usepackage[ngerman]{babel}
\usepackage{lscape}
\usepackage{cancel}
\usepackage{	}
\usepackage{listings}

\definecolor{codegreen}{rgb}{0,0.6,0}
\definecolor{codegray}{rgb}{0.5,0.5,0.5}
\definecolor{codepurple}{rgb}{0.58,0,0.82}
\definecolor{backcolour}{rgb}{0.95,0.95,0.92}
\lstdefinestyle{mystyle}{
    backgroundcolor=\color{backcolour},   
    commentstyle=\color{codegreen},
    keywordstyle=\color{magenta},
    numberstyle=\tiny\color{codegray},
    stringstyle=\color{codepurple},
    basicstyle=\ttfamily\footnotesize,
    breakatwhitespace=false,         
    breaklines=true,                 
    captionpos=b,                    
    keepspaces=true,                 
    numbers=left,                    
    numbersep=5pt,                  
    showspaces=false,                
    showstringspaces=false,
    showtabs=false,                  
    tabsize=2
}

\lstset{style=mystyle, language = java}

\begin{document}
\begin{titlepage}
    \centering
    {\scshape\LARGE Hochschule für Technik und Wirtschaft Dresden \par}
    \vspace{1cm}
    {\scshape\Large Berechnung der Mandelbrotmenge \\mit Implementierung von Java-RMI \par}
    \vspace{1.5cm}
    {\huge\bfseries Development Document\par}
    \vspace{2cm}
    {\Large\itshape David Kirchner, Maxim Haschke, Tan Minh Ho, Quang Duy Pham\par}
    \vfill

    {\large \today\par}
\end{titlepage}

\tableofcontents

\newpage
\section{Kurzbeschreibung an Belegarbeit}
	Im Rahmen des Moduls Verteilte Systeme, soll als Beleg eine Java Anwendung zur Berechnung und Visualisierung der Mandelbrot-Menge entwickelt werden. Dabei sollen folgende Punkte umgesetzt werden:
	\begin{itemize}
		\item schrittweiser Zoom zu einem bestimmten Bildpunkt
		\item Nutzung von Java-RMI oder einer anderen geeigneten Technologie zur Verteilung der Aufgaben
		\item Architektur Model-View-Presenter für den Client
		\item Dokumentation der Programmstruktur
	\end{itemize}

\section{Architektur}
	Nach der Diskussion mit anderen Gruppenteilnehmer wird das Programm mit folgenden Merkmalen realisiert:
	\begin{itemize}
		\item Übertragung mittels Java-RMI, basierend auf TCP
		\item Nutzung eines 2-dimensionalen int Arrays zur Übertragung der RGB Pixelfarben
		\item Thread-Pool im Client zur beschleunigten Konvertierung von int Werte in BufferedImage Bildpunkte
		\item Thread-Pool im Server zur gleichzeitigen und beschleunigten Berechnung der Bildpunkte
	\end{itemize}
	
\section{Details der Programme}
\subsection{Client}
	Für den Client gibt es drei Java-Dateien mit folgenden Bezeichnungen:
	\begin{itemize}
		\item MandelClient.java
		\item MandelClientImpl.java
		\item MandelClientMain.java
	\end{itemize}
\subsubsection{MandelClient.java}
	Die Datei \glqq MandelClient.java\grqq definiert die Schnittstelle für das Clientprogramm. Die Methoden setRGB, sendTasks und setZoomDestination sind hier als Remote mit RemoteException deklariert.
	\lstinputlisting[]{../src/MandelClient.java}

	\newpage
\subsubsection{MandelClientImpl.java}
	Die Datei \glqq MandelClientImpl.java\grqq implementiert die oben genannte Methoden für den Client.\\

	\begin{enumerate}
		\item Der Konstruktor setzt die notwendigen Bildwerte zum Server und setzt für und ActionListener fest.\\
			\lstinputlisting[firstline=43, lastline=107]{../src/MandelClientImpl.java}
		\item sendTasks schickt ein Request zum Server und wartet auf dessen Antwort\\
			\begin{enumerate}
				\item Senden der Koordinaten und Werte von ActionListener zum Server
				\item Server startet Berechnung der Mandelbrotmenge mit Zoom zu den gegebenen Koordinaten
				\item Client wartet auf Server
				\item Client zeigt die zurückgegebenen RGB-Werte von Server als BufferedImage an.
			\end{enumerate}
			\lstinputlisting[firstline=109, lastline=122]{../src/MandelClientImpl.java}
		\item setRGB setzt die Pixelfarbe des Bildes mit Nutzung eines Thread-Pools.
			\lstinputlisting[firstline=197, lastline=207]{../src/MandelClientImpl.java}
	\end{enumerate}
	
	\newpage
	\subsubsection{MandelClientMain.java}
	Die Datei \glqq MandelClientMain.java\grqq sucht den Server (RMI-Registry) mittels IP/ Hostnamen (welcher durch die args[0]-Variable gesetzt wurde). Durch die args[1]-Variable wird das Name Binding des Servers spezifiziert. Anschließend sendet der Client mit sendTasks() die Berechnungsaufgabe an den Server.
	\lstinputlisting[]{../src/MandelClientMain.java}

	\newpage
	\subsection{Server}
	für den Client gibt es drei Javadateien mit Name:
	\begin{itemize}
		\item MandelServer.java
		\item MandelServerImpl.java
		\item MandelServerMain.java
	\end{itemize}
	
	\subsubsection{MandelServer.java}
	Die Datei \glqq MandelServer.java\grqq definiert die Schnittstelle für das Serverprogramm. Die Methoden setDetail, setImageProperties, calculateRGB, returnColor, startCalculatingRGB und isFinish sind hier als Remote mit RemoteException deklariert.
	\lstinputlisting[]{../src/MandelServer.java}
	
	\newpage	
	\subsubsection{MandelServerImpl.java}
	\begin{enumerate}
		\item Konstruktor definiert Thread-Pool für Parallelität des Servers
			\lstinputlisting[firstline=24, lastline=42]{../src/MandelServerImpl.java}
		\item setDetails definiert die Laenge, Breite und Details (Iterationen) des Bildern und deklariert ein 2-dimensionales Array mit passender Laenge und Breite.
			\lstinputlisting[firstline=65, lastline=70]{../src/MandelServerImpl.java}
		\item setImageProperties setzt die Koordinaten von MouseActionListener.
			\lstinputlisting[firstline=72, lastline=76]{../src/MandelServerImpl.java}
		\item returnColor schickt ein 2-dimensionales Array zum Client zurueck
			\lstinputlisting[firstline=78, lastline=80]{../src/MandelServerImpl.java}
		\newpage		
		\item startCalculateingRBG schnitt das Bild zu kleineren Bilden mit Hilfe der Nummer des Threads im Thread-Pool. Falls alle Threads "busy" sind, wartet Segment des Bildes 
			\lstinputlisting[firstline=82, lastline=100]{../src/MandelServerImpl.java}
		\item isFinish beanwortet, ob Bild fertig bearbeitet hat
			\lstinputlisting[firstline=102, lastline=107]{../src/MandelServerImpl.java}
		\item calculateRGB stellt Aufgabe zum Thread fest, um parallele Berechnung zu erreichen
			\lstinputlisting[firstline=109, lastline=112]{../src/MandelServerImpl.java}
		\newpage
		\item Task berechnet die Mandelbrotmenge
			\lstinputlisting[firstline=114, lastline=149]{../src/MandelServerImpl.java}
	\end{enumerate}
	
	\subsection{MandelServerMain.java}
	Die Datei \glqq MandelServerMain.java\grqq stellt ein Registry fest und verbindet mit Serverschnittstelle
	\lstinputlisting[]{../src/MandelServerMain.java}
	
	\subsubsection{Multi Server Funktionalität}
	Unser ursprünglicher Plan war es, die Möglichkeit zu bieten, die Berechnungen nicht nur auf einen entfernten Rechner
	mit Multithreading auszulagern, sondern die Berechnung nochmals auf mehrere Server aufzuteilen.
	Die Idee war dabei, dass der Client mit einem Master Server kommuniziert, welcher wiederum mehrere Slave Server koordiniert.
	Bei z.B. 3 Slave Servern sollte das Bild in 3 Zeilen zerlegt werden, wobei jeder Slave Server je ein Drittel des gesamten Bildes
	berechnet hätte.
	Auf jedem Slave Server wären dann die zugeteilten Bildzeilen nochmals in n Teile zerlegt worden, um damit n Threads mit Multi-Threading
	zu starten.\\
	Leider war es uns zeitlich nicht mehr möglich diese Funktion fertigzustellen.
	Der bisherige Stand lässt sich im Branch multi-server einsehen.
	
	\section{Bedienung}
	Im Client kann manuell im Bild mithilfe der Maus navigiert werden. Ein Linksklick zoomt das Bild auf die aktuelle Position des Mauszeigers hinein.
	Mit Rechtsklick kann wieder herausgezoomt werden.\\
	An der Rechten Seite befinden sich 4 Text Felder in denen die x/y Position für den Startpunkt und den Zielpunkt der Animation eingetragen werden
	können. Mit einem Klick auf den Start Button, beginnt die Animation.
	
	
	
	\newpage
	\section{Link und Literatur}
	\begin{itemize}
		\item Parallele und verteilte Anwendungen in Java, Rainer Oechsle
		\item Berechnung der Mandelbrotmenge: https://www.youtube.com/watch?v=0bgrzmtu4e8
	\end{itemize}
\end{document}


